\chapter*{Abstract}
\defineHeaderForStarredChapter[Abstract]

\vspace{-1.3cm}

Augmented Reality (AR) consists in inserting virtual elements in a real scene, observed through a screen or a projection system on the scene or the object of interest. The augmented reality systems can take different forms to obtain a balance between three criteria: precision, latency and robustness. It is possible to identify three main components to these systems: \textit{localization, reconstruction and display}. The contributions of this thesis focus essentially on the display and more particularly the rendering of augmented reality applications.

Contrary to the recent advances in the field of localization and reconstruction, the insertion of virtual elements in a plausible and aesthetic way remains a complicated problematic, ill-posed and not adapted to a real-time context. Indeed, this insertion requires a good understanding of the lighting conditions of the scene. The lighting conditions of the scene can be divided in several categories. First, we can model the environment to describe the interaction between the incident and reflected light pour each 3D point of a surface. Secondly, it is also possible to explicit the environment by computing the position of the light sources, their type (desktop lamps, fluorescent lamp, light bulb, \ldots), their intensities and their colors. Finally, To insert a virtual object in a coherent and realistic way, it is essential to have a knowledge of the surface's geometry, its chemical composition (material) and its color. For all of these aspects, the reconstruction of the illumination is difficult because it is really complex to isolate the illumination without prior knowledge of the geometry, material of the scene and the camera pose observing the scene.

In general, on a surface, a light source leaves several traces such as shadows, created from the occultation of light rays by an object, and the specularities (or specular reflections) which are created by the partial or total reflection of the light. These specularities are often described as very high intensity elements in the image.

Although these specularities are often considered as outliers for applications such as camera localization, reconstruction or segmentation, these elements give crucial information on the position and color of the light source but also on the surface's geometry and the material's reflectance where these specularities appear. To address the light modeling problem, we focused, in this thesis, on the study of specularities and on every information that they can provide for the understanding of the scene. More specifically, we know that a specularity is defined as the reflection of the light source on a shiny surface. From this statement, we have explored the possibility to consider the specularity as the image created from the projection of a 3D object in space.

%Cette considération est à l'opposé de nombreuses méthodes de l'état de l'art qui formalisent les phénomènes lumineux comme étant majoritairement photométriques.

We started from the simple but little studied in the literature observation that specularities present an elliptic shape when they appear on a planar surface. From this hypothesis, can we consider the existence of a 3D object fixed in space such as its perspective projection in the image fit the shape of the specularity? We know that an ellipsoid projected perspectivally gives an ellipse. Considering the specularity as a geometric phenomenon presents various advantages. First, the reconstruction of a 3D object and more specifically of an ellipsoid, has been to many publications in the state in the art.
%De plus, reconstruire une quadrique est un procédé en général plus simple et plus rapide à l'opposé des méthodes de reconstruction de l'illumination classiques qui prennent en considération l'intégralité de l'image et requiert de nombreux processus de minimisation non-linéaire pour de nombreuses variables.
Secondly, this modeling allows a great flexibility on the tracking of the state of the specularity and more specifically the light source. Indeed, if the light is turning off, it is easy to visualize in the image if the specularity disappears if we know the contour (and reciprocally of the light is turning on again). By considering the problem of light modeling in a photometric way, it is way more difficult to separate the specular component from the rest of the scene because the models from the state of the art does not explicit in general the specular part. Moreover, a specularity is often in an explicit and detailed form in a video sequence. An important quantity of information can be recovered by studying the shape of the specularity without prior knowledge of the material or roughness of the surface.

Firstly, we have formalized a geometric model dedicated to the specularity prediction called \textit{JOint LIght-MAterial Specularity} (JOLIMAS). The formalism and the reconstruction of this model have been studied for the case of planar surfaces first. Afterwards, we extended our model to convex surfaces in order to improve the genericity of our model in a new model called dual JOLIMAS. This improvement has been realized by using a virtual representation of a camera which allows to reconstruct the model in a more generic and flexible way. Finally, we have generalized the \mbox{JOLIMAS} model for every curvature by studying the link between the curvature changes and the changes in the specularity shape. More specifically, for a specularity observed on a surface of any curvature, we modify the contours associated to the specularity in order to simulate the shape that the specularity would have specularity on a planar surface (non-curved). By applying this transformation for each specularity observed, we can reconstruct our JOLIMAS model in a canonical representation of the quadric independent of the curvature of the surfaces where the specularities are observed. For the specularity prediction, we do the inverse process by projection, in the form of an ellipse shape, the quadric according to the camera pose which give the specularity's shape for a planar surface. We then modify the contours of projected ellipse to match the curvature of the surface and fit the shape of the specularity for a new viewpoint.
Note that for every versions of JOLIMAS, the camera poses and the geometry of the surface, where the specularities is present, are known. Moreover, we suppose that the reflectance of the surfaces used to reconstruct our model is constant.   To show the application scope of the model, we have developed a rexturing application which allows to change the texture of a specular surface and to synthesize the specularity that has been removed from the new texture by using the JOLIMAS model. Since our model only provides the external contour of a specularity, we proposed new intensity functions to synthesize the photometric behavior of a specularity.

This thesis has been the object of several international publications: VISAPP \citeyearpar{morgand2014generic}, ISMAR \citeyearpar{morgand2016empirical, morgand2017multipleview}, two international TVCG journals \citeyearpar{morgand2017amultiple, morgand2017multiple}, four national conferences: two ORASIS articles \citeyearpar{morgand2015reconstruction, morgand2015detection}, RFIA \citeyearpar{morgand2015modele} and ORASIS \citeyearpar{morgand2017modele} and a patent accepted in June \citeyearpar{tamaazousti2018method}. A submission to a TVCG journal is under review at tje time of the writing of the manuscript.

\paragraph{Key-words:} Specularity prediction, Augmented Reality, Retexturing, Quadric reconstruction, Conic, Illumination, Multiple views.

